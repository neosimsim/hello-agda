\documentclass[12pt]{article}

\usepackage{neosimsim}
\usepackage{neosimsim-a4paper}

\usepackage[backend=biber]{biblatex}
\addbibresource{bibliography.bib}

\newcommand{\ignore}[1]{}
\usepackage{fancyvrb}

\DefineVerbatimEnvironment%
  {code}{Verbatim}
  {frame=single} % Add fancy options here if you like.

\DefineVerbatimEnvironment%
  {listing}{Verbatim}
  {frame=single} % Add fancy options here if you like.

\begin{document}
\section{Installation}
I feel Agda a quite hard to setup. Especially compared to Haskell. This is even
more true if you use Stack and you have to do is call \verb+wget+ once.

\subsection{Agda}
The installation of \verb+agda+ turned out to be tricky.
When I tried to install \verb+agda+ for the first time I just used \verb+cabal+
instead \verb+stack+. I clearly recommend to use \verb+stack+ now.

\begin{lstlisting}
cabal v2-install Agda
stack install Agda
\end{lstlisting}

The package \emph{Agda} installs to binaries

\begin{description}
	\item[agda] The Agda compiler
	\item[agda-mode] An Agda emacs front-end

		I won't go into \verb+agda-mode+ since I'm a vim guy.
\end{description}

You can also check your systems package manager.
\begin{lstlisting}
pacman -S agda
apt-get install agda-mode
\end{lstlisting}

For further information, please have a look at the Agda User
Guide~\cite{agda-user-guide}.

Using Stack you might experience problems with extra-deps. As a personal
recommendation I use the following script to invoke \verb+agda+:
\lstinputlisting[caption=agda]{agda}
Keep this file some where in \verb+$PATH+ as \verb+agda+.

\subsection{Agda StdLib}
Agda has a some built-in modules, but you want at least to install
the agda-stdlib~\cite{agda-stdlib} to really have fun with Agda.

Like any good programming language, Agda has no really development
infrastructure. There are two more or less convenient ways to install
libraries for Agda.

\begin{description}
	\item[Manually] Download the standard library~\cite{agda-stdlib}.
		\begin{lstlisting}[caption={\$HOME/.agda/libraries}]
/home/neosimsim/src/agda/std-lib
		\end{lstlisting}
		\begin{lstlisting}[caption={\$HOME/.agda/defaults}]
standard-library
		\end{lstlisting}
	\item[agda-pkg] An Agda package manager written in python.
		\begin{lstlisting}
pip install agda-pkg
apkg init
apkg install standard-library
		\end{lstlisting}

		Keep in mind that it's a Python tool and might break after some update
		due to dynamic dependencies. It's a scripting language after all.
\end{description}

\subsection{VIM}

Agda makes heavy use of mathematical symbols, e. g. ℕ and ℤ. So you should be able
to type them in your editor. For vim you can find proper vim-scripts on the
Agda wiki page~\cite{vim-editing}.

Here are snippets of the files.
\lstinputlisting[caption={.vim/ftplugin/agda.vim},lastline=10]{agda.vim}
\lstinputlisting[caption={.vim/ftplugin/unicode-keys.vim},lastline=10]{unicode-keys.vim}
Safe them under the proper location.

Then you have to tell vim how to detect Agda files.
\begin{lstlisting}[caption={.vim/ftdetect/agda.vim}]
autocmd BufNewFile,BufReadPost *.agda set filetype=agda
autocmd BufNewFile,BufReadPost *.lagda.md set filetype=agda
autocmd BufNewFile,BufReadPost *.lagda.tex set filetype=agda
\end{lstlisting}

A more general way is provided by a command line tool I wrote.

\begin{listing}
echo "\N" | uni
ℕ
\end{listing}

This can be used from withing vim, e. g.\ in visual line mode and typing
\begin{listing}
:'<,'>!uni
\end{listing}

I you still use X11 you can also create the file
\lstinputlisting[caption={~/.Xmodmap}]{Xmodmap}
and call
\begin{listing}
xmodmap ~/.Xmodmap
\end{listing}
Then you can type ℕ by pressing AltGr-n.

\section{Hello World}
\lstinputlisting[caption={hello-world.agda}]{hello-world.agda}

\section{Coming From Haskell}
\subsection{Unicode}
boo
\ignore{
\begin{code}
module unicode where

open import IO
open import Data.Bool
\end{code}
}

\begin{code}
if_☺_☹_ : {A : Set} -> Bool -> A -> A -> A
if true ☺ x ☹ y = x
if false ☺ x ☹ y = y

main = if true
  ☺ (run (putStrLn "Hallo, Welt!"))
  ☹ (run (putStrLn "Hello, World!"))
\end{code}

\subsection{Mixfix}
\section{}

\printbibliography{}
\end{document}
