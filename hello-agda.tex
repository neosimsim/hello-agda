\documentclass[12pt]{article}

\usepackage{neosimsim}
\usepackage{neosimsim-a4paper}

\usepackage[backend=biber]{biblatex}
\addbibresource{bibliography.bib}

\usepackage{titlesec}
\newcommand{\sectionbreak}{\clearpage}

\usepackage{fancyvrb}
\newcommand{\ignore}[1]{}

\newtheorem{exercise}{Exercise}

\DefineVerbatimEnvironment%
  {code}{Verbatim}
  {} % Add fancy options here if you like.

\DefineVerbatimEnvironment%
  {neolisting}{Verbatim}
  {frame=single} % Add fancy options here if you like.

% https://tex.stackexchange.com/questions/18322/using-fancyvrb-verbatim-environment-from-within-newenvironment
\usepackage{fancyvrb,caption,floatrow}
\DeclareNewFloatType{neolistingfloat}{placement=t,fileext=neolisting,name=Listing}
\captionsetup[neolistingfloat]{}

\newenvironment{neocaplisting}[2]
{\VerbatimEnvironment%
 \captionof{neolistingfloat}{#2}\ifx\relax#1\relax\else\label{#1}\fi%
\begin{Verbatim}[frame=single]}
{\end{Verbatim}}

\begin{document}

\tableofcontents{}

I suggest we limit this workshop to 2 hours. This shall be no lecture nor a school.
I want to make hits interactive and every should take avay something. So I suggest
we'll take all the time need and contiue where we left next time. OK?

\begin{itemize}
\item Who has used Agda before
\item Who has used a dependent typed programming language before
  \item Who used a proof system before
\end{itemize}

\input{installation}
\input{hello-world}
\section{Natural Numbers}
Now that we can write code in Agda, our next goal to understand why we
want to write code in Agda, i. e. we want to write code that can’t be
written in other languages. Not even in Haskell—well not yet, or at
least not in an easy way.

I mean of course the best feature a programming language can have
\emph{Dependent Types}. The standard example for Dependent Types is the
fix length vector. Don’t worry, if you don’t know what I mean by that
you will soon.

Before we can describe fix length vectors we have to be able to describe
length. What better way to describe length but with natural numbers.

\begin{code}
module Naturals where

data ℕ : Set where
  zero : ℕ
  suc  : ℕ → ℕ
\end{code}

What's that? This is an algebraic data type similar to those we wrote and use
in Haskell all day long, or at least we would if we could use Haskell all day
long at work.

\begin{haskell}
data Nat = Zero | Suc Nat
\end{haskell}

If we write the same with Haskells GADTs extension we see the resemblance:

\begin{haskell}
{-# LANGUAGE GADTs #-}

data Nat where
  Zero  :: Nat
  Suc   :: Nat -> Nat
\end{haskell}

So far so good, but what about \verb+Set Whe state that \verb+Bool+ is of type
\verb+Set+, which is the type of small types. In Haskell we would say of kind `*`.
\verb+Set+ itself is of type \verb+Set1+ which is of type \verb+Set2+ and so on. In Haskell
we would say nothing like this.

\begin{haskell}
{-# LANGUAGE GADTs          #-}
{-# LANGUAGE KindSignatures #-}

data Bool :: * where
  Zero  :: Nat
  Suc   :: Nat -> Nat
\end{haskell}

In other word being a Type in Haskell means being a Set in Agda.

That's all we need, to describe every natural number possible.
Quite cool when you think about it. No little-endian, no big-endian.

Let's define some natural numbers:
\begin{code}
null : ℕ
null = zero

three : ℕ
three = suc (suc (suc zero))
\end{code}

\begin{exercise}
  Define \verb+fivethousand+.
\end{exercise}

GOTCHA!!!

We can tell Agda to use Haskell builtin Integers as \verb+ℕ+.

\begin{code}
{-# BUILTIN NATURAL ℕ #-}

fivethousand : ℕ
fivethousand = 5000000
\end{code}

Let's define additions.
\begin{code}
infixl 6 _+_
_+_ : ℕ → ℕ → ℕ
\end{code}

Using \verb+_+ in function names we define pre/in/mix/post-fix operators. The
$n$-th arguments takes place of the $n$-th \verb+_+.
\begin{code}
zero + m = m
suc n + m = suc (n + m)
\end{code}

Since we are all good programmers we want to test this function.
We need some imports.
\begin{code}
import Relation.Binary.PropositionalEquality as Eq
open Eq using (_≡_; refl; cong; sym)
open Eq.≡-Reasoning using (begin_; _≡⟨⟩_; _≡⟨_⟩_; _∎)
\end{code}

With this we can write a unit test for \verb=_+_=.
\begin{code}
_ : 2 + 3 ≡ 5
_ = refl
\end{code}
For now (this little moment in time) you can think of \verb+refl+ as
of an assertion.

We can write the test in a more human readable form.
\begin{code}
_ : 2 + 3 ≡ 5
_ =
  begin
    2 + 3
  ≡⟨⟩
    suc (1 + 3)
  ≡⟨⟩
    suc (suc (0 + 3))
  ≡⟨⟩
    suc (suc 3)
  ≡⟨⟩
    5
  ∎
\end{code}

This is a lot to take in. First we notice that \verb=2 + 3 ≡ 5= is a type signature.
And \verb+_≡_+ is a type constructor.
\begin{verbatim}
infix 4 _≡_
data _≡_ {a} {A : Set a} (x : A) : A → Set a where
  instance refl : x ≡ x
\end{verbatim}
\verb+x ≡ x+ is the type of elements which are equal to itself, hence \emph{reflexive}.

So
\begin{verbatim}
_ : 2 + 3 ≡ 5
\end{verbatim}
defines a variable \verb+_+ containing the information that $2+3$ is equal to $5$.
The variable name \verb+_+ states, that we don't care about the variable itself.

\begin{verbatim}
_ = refl
\end{verbatim}
tells Agda to construct an inhabitant of the type \verb=2 + 3 ≡ 5=, if it can
we know they are equal.

\begin{exercise}
Write a test for $3+4$.
\end{exercise}

Unlike humans, Agda does not need every step. In fact everything Agda
does is to check, that every expression between \verb+≡⟨⟩+ reduces to the
same expression. Since Agda can reduce the expressions without further
information we could just use \verb+refl+.

\subsection{Proofs}
With that in mind we could say that we did not just write a test $2+3$ but a proof.
The cool thing is, that Agda verifies the proof at compile time, so no need to
actually “run the test”.

We just experienced the phenomenon of “Proposition as Types” — whooooo

If course this proof is as useful as any other unit test … not at all.
We test addition for 2 natural numbers this is a fraction of 0\,\%.

We could use QuickCheck to run property test. But again 0\,\% coverage.

In Agda we can do better but first we need some properties for addition to fulfill:
\begin{itemize}
  \item $0$ is the neutral element
  \item assosiative
  \item commutative
\end{itemize}

First we show that \verb+zero+ is the neutral element. We already
know that \verb+zero+ is neutral rom the left by definition.
\begin{code}
+-rightNeutral : ∀ (m : ℕ) → m + zero ≡ m
+-rightNeutral zero =
  begin
    zero + zero
  ≡⟨⟩
    zero
  ∎
+-rightNeutral (suc m) =
  begin
    suc m + zero
  ≡⟨⟩
    suc (m + zero)
  ≡⟨ cong  suc (+-rightNeutral m) ⟩ -- congruent
    suc m
  ∎
\end{code}
Agda makes it explicit that proof by induction is the same a programming
with recursion.

\begin{exercise}
We defined \verb=_+_= left inductive. Proof that it \verb=_+_= is to right inductive, too.
I. e. proof

\begin{code}
+-suc : ∀ (m n : ℕ) → m + suc n ≡ suc (m + n)
\end{code}
\end{exercise}

\begin{code}
+-suc zero n =
  begin
    zero + suc n
  ≡⟨⟩
    suc n
  ≡⟨⟩
    suc (zero + n)
  ∎
+-suc (suc m) n =
  begin
    suc m + suc n
  ≡⟨⟩
    suc (m + suc n)
  ≡⟨ cong suc (+-suc m n) ⟩
    suc (suc (m + n))
  ≡⟨⟩
    suc (suc m + n)
  ∎
\end{code}

Whit that lemma we can proof commutativity.
\begin{code}
+-commutative : ∀ (n m : ℕ) → n + m ≡ m + n
+-commutative zero m =
  begin
    zero + m
  ≡⟨⟩
    m
  ≡⟨ sym (+-rightNeutral m) ⟩
    m + zero
  ∎
+-commutative (suc n) m =
  begin
    suc n + m
  ≡⟨⟩
    suc (n + m)
  ≡⟨ cong suc (sym (+-commutative m n)) ⟩
    suc (m + n)
  ≡⟨ sym (+-suc m n) ⟩
    m + suc n
  ∎
\end{code}

\begin{exercise}
Show that $+$ is assosiative, i.e.

\begin{code}
+-associativ : ∀ (l : ℕ) (m : ℕ) (n : ℕ)
  → (l + m) + n ≡ l + (m + n)
\end{code}
\end{exercise}

\begin{code}
+-associativ zero m n =
  begin
    (zero + m) + n
  ≡⟨⟩
    m + n
  ≡⟨⟩
    zero + (m + n)
  ∎
+-associativ (suc l) m n =
  begin
    ((suc l) + m) + n
  ≡⟨⟩
    (suc (l + m)) + n
  ≡⟨⟩
    suc ((l + m) + n)
  ≡⟨ cong suc (+-associativ l m n) ⟩ -- congruent
    suc (l + (m + n))
  ≡⟨⟩
   suc l + (m + n)
  ∎
\end{code}

Now we have tests for addition with 100\,\% coverage over the input domain,
also called \emph{proofs}.

Maybe we should stop here.

\begin{code}
infixl 7 _*_
_*_ : ℕ → ℕ → ℕ
zero * _ = zero
suc m * n = n + (m * n)
\end{code}

\begin{exercise}
Proof that $*$ distributive on $+$, .i.e.

\begin{code}
*-distributive : ∀ (l : ℕ) (m : ℕ) (n : ℕ)
  → l * (m + n) ≡ l * m + l * n
\end{code}
\end{exercise}

\begin{code}
*-distributive zero m n =
  begin
    zero * (m + n)
  ≡⟨⟩
    zero
  ≡⟨⟩
    zero * zero
  ≡⟨⟩
    zero * m + zero * n
  ∎
*-distributive (suc l) m n =
  begin
    (suc l) * (m + n)
  ≡⟨⟩
    (m + n) + (l * (m + n))
  ≡⟨ cong ((m + n) +_) (*-distributive l m n) ⟩
    (m + n) + (l * m + l * n)
  ≡⟨ +-associativ m n (l * m + l * n) ⟩
    m + (n + (l * m + l * n))
  ≡⟨ cong (m +_) (sym (+-associativ n (l * m) (l * n))) ⟩
    m + ((n + l * m) + l * n)
  ≡⟨ cong (m +_) (cong (_+ l * n) (+-commutative n (l * m))) ⟩
    m + ((l * m + n) + l * n)
  ≡⟨ cong (m +_) (+-associativ (l * m) n (l * n)) ⟩
    m + (l * m + (n + l * n))
  ≡⟨ sym (+-associativ  m (l * m) (n + l * n)) ⟩
    (m + l * m) + (n + l * n)
  ≡⟨⟩
    suc l * m + suc l * n
  ∎
\end{code}


\section{Coming From Haskell}
\subsection{Unicode}
\ignore{
\begin{code}
module unicode where

open import IO
open import Data.Bool
\end{code}
}

\begin{code}
if_☺_☹_ : {A : Set} -> Bool -> A -> A -> A
if true ☺ x ☹ y = x
if false ☺ x ☹ y = y

main = if true
  ☺ (run (putStrLn "Hallo, Welt!"))
  ☹ (run (putStrLn "Hello, World!"))
\end{code}

\subsection{Mixfix}

\printbibliography{}
\end{document}
